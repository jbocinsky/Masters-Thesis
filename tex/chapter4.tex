\chapter{RESULTS} \label{results}

\section{Data}

\subsection{Land Mine Detection}
What is the goal of land mine detection?

\subsection{Data Collection 1}
Explain where and when the data was collected...\newline
Explain what a sweep is and the different variations of sweeps...\newline
Show what each variation looks like plotted out over UTM coordinates

\subsubsection{Sensors Used}
Explain the types of hand held sensors that were used to collect data

\subsubsection{Land Mine Types}
What are the different types of land mines that are studied?\newline
High Metal vs Low Metal

\subsubsection{Data Format}
Explain what a lane is...\newline
Explain what a grid is...




\section{Experiments}

\subsection{Experiment 1: Clustering Methods}
Answer the following: Using a clustering initialization approach to the multi-target MIACE algorithm, which of these three clustering methods (clustering1, clustering2, and clustering3) performs the best, considering ability and efficiency?

\subsubsection{Ability}


\subsubsection{Efficiency}
Run Time Comparison \newline
Space Comparison \newline
Data Awareness Comparison - How much prior knowledge about the data is required before running an algorithm\newline


\subsection{Experiment 2: Multi-Target MIACE Comparison}
Answer the following: Between the 3 variations of the multi-target MIACE algorithm: greedy approach, thresholding approach, and clustering approach, is there a variation that is able to choose the underlying EMI target signatures? If so, which method performs the best considering ability and efficiency?

\subsubsection{Ability}

\subsubsection{Efficiency}
Run Time Comparison \newline
Space Comparison \newline
Data Awareness Comparison - How much prior knowledge about the data is required before running an algorithm\newline


\subsection{Experiment 3: Related Methods Comparison}
Answer the following: Are any of the three types of multi-target MIACE algorithms better, considering ability and efficiency, than a single MIACE algorithm on individual target type classes and/or Global ACE using a generated Discrete Spectrum of Relaxation Frequencies (DSRF)?

\subsubsection{Ability}

\subsubsection{Efficiency}
Run Time Comparison \newline
Space Comparison \newline
Data Awareness Comparison - How much prior knowledge about the data is required before running an algorithm\newline











